%! Author = ben
%! Date = 25/10/2021

% Preamble
\documentclass[11pt]{article}
\usepackage[top=2.54cm, bottom=2.54cm, left=2.75cm, right=2.75cm]{geometry}

% Packages
\usepackage{amsmath}

% Document
\begin{document}

    \section*{Login Details for experiementing account}
    Email: space.in.the.news@gmail.com
    \\
    Passsword: e[G`uaX?9i"aieJ

    \subsection{Users}
    Username: Ben
    \\
    Access Type: AWS Managament Console access
    \\
    Console Password: sdjf248r+(23ef)-af2+
    \\
    Sign in link: https://266867367924.signin.aws.amazon.com/console

    \section{Accroynyms}
    \begin{itemize}
        \item CIDR = Classless Interdomain Routing
        \item VPC = Virual Private Cloud
        \item NACLs = Network Access Control Lists
    \end{itemize}

    \section{Basics and Introduction}
    \subsection{AWS Global Infrastructure}
    \begin{itemize}
        \item Availability zone = one or more data centres
        \item Each region consists of two or more Availability zones.
        \item Each region is completely independent.
        \item Every region is connected via a high bandwidth, fully redundant network. (does not use the internet - more consistent result than internent)
    \end{itemize}

    \subsection{AWS Pricing}
    \subsubsection{Compute}
    Amounnt of resources (eg. CPU, RAM, \ldots)

    \subsubsection{Storage}
    Quantity of data (size allocated for data normally)


    \subsubsection{Outbound Data Transfer}
    Don't pay for inbound data, but do pay for the quantity of data that is transferred out from all services.

    \section{Create Billing Alarm}
    \begin{enumerate}
        \item Navigate to \textbf{Billing \& Cost Management Dashboard} from the main page (can be searched for (search for 'billing')).
        \item select \textbf{preferences} on the left
    \end{enumerate}
    Can also (from the main page (services drop down)) go to CloudWatch.
    \\
    \begin{enumerate}
        \item Go to \textbf{CloudWatch}
        \item Change region to N. Virginia
        \item select \textbf{In Alarm}
        \item create alarm
        \item billing alarm
        \item total
        \item leave settings the same
        \item click next to set up SNS \textbf{Simple Notification Service}
        \item Select \textbf{In Alarm}
        \item Create new topic
        \item Set topic and email
        \item set up topic
        \item confirm email subscription
        \item can  be viewed in the SNS Console
        \item Click next
        \item set alarm name
        \item next
        \item click \textbf{create alarm}
        \item done
    \end{enumerate}

    \section{AWS Identity and Access Management Service (IAM)}
    Can apply an \textbf{IAM Policy} to an \textbf{IAM User} or \textbf{IAM Group} (to define permissions)
    \\
    \textbf{IAM User} = an entity that represents a person or service.
    \\
    \textbf{IAM Group} = a collection of users and have poilicies attached to them.
    \\
    \textbf{IAM Role} = Roles are "assumed" by trusted entities and can be used for delegation.
    \\
    \textbf{Access Key} = Consists of an Access key ID and a secret access key.
    Used for programmatic access to the API\@.

    \section{Create IAM User and Group}
    \begin{enumerate}
        \item Select \textbf{Services} drop down in the main bar
        \item select \textbf{IAM} under the \textbf{Security, Identity, \& Compliance} section.
        \item
    \end{enumerate}
    \subsection{Create a User Group}
    \begin{enumerate}
        \item click on \textbf{User Groups} (or the number below it)
        \item Click \textbf{Create Group}
        \subitem (blue button, top right)
        \item Set Group name
        \item Attach Permissions Policy.
        \subitem for full administrator access, select \textbf{AdministratorAccess}
        \item Click \textbf{Create Group}
        \subitem (blue button, bottom right)
    \end{enumerate}
    \subsubsection{Create User (to be added to the user group)}
    \begin{enumerate}
        \item Under \textbf{Access management} on the LHS panel, select \textbf{Users}.
        \item Click \textbf{Add User} (blue button, top right).
        \item Type in a user name
        \item select access type and provide a password to be used by user if necessary.
        \item click next
        \subitem ... onto permisions
        \item Add User to the existing \textbf{Admins} group.
        \item Once confirmed, can sign in using the link provided (using the username and password just created)
    \end{enumerate}

    %%%%%%%%%%%%% Amazon Virtual Private Cloud (VPC) %%%%%%%%%%%%%
    \section{Amazon Virtual Private Cloud (VPC)}
    VPC = a logically isolated portion of the AWS cloud within a region.
    \\
    Subnets are created within availability zones (AZs).
    \\
     - Can have Public and Private subnets
    \\
     - Public subnets means instances running in it get a public IP address and can access the internet gateway directly.
    \\
     - Private subnets cannot access the internet gateway directly.
    (there is a way to get around this though)
    \\
    Route Table is used to configure the VPC router.
    \\
    Route Tables can be assigned to the subnets.
    \\
    Can launch virtual servers into VPC subnets.
    \\
    Each VPC has a different block of IP addresses.
    \\
    CIDR = Classless Interdomain Routing.
    (IP addressing concept)
    \\
    Can Create multiple VPCs within each region.
    \\
    Each subnet has a block of IP addresses from the CIDR block.
    \\\\
    To view VPCs, go to Services, VPC (under networking and content delivery)

    %%%%%%%%%%%%% Security Groups \& Network Access Control Lists (NACLs) %%%%%%%%%%%%%
    \section[NACLs]{Security Groups \& Network Access Control Lists (NACLs)}
    \begin{itemize}
        \item NACLs = Types of Firewalls
        \subitem NACLs Apply at the subnet level.
        \item Security Groups apply at the instance level
        \subitem applies to instance level communication (within security group)
        \subitem can be applied to instances in any subnet
    \end{itemize}


\end{document}