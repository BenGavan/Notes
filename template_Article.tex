\documentclass[]{article}

\usepackage{color}
\definecolor{editorGray}{rgb}{0.95, 0.95, 0.95}
\definecolor{editorOcher}{rgb}{1, 0.5, 0} % #FF7F00 -> rgb(239, 169, 0)
\definecolor{editorGreen}{rgb}{0, 0.5, 0} % #007C00 -> rgb(0, 124, 0)
\usepackage{upquote}
\usepackage{listings}
\lstdefinelanguage{JavaScript}{
	morekeywords={typeof, new, true, false, catch, function, return, null, catch, switch, var, if, in, while, do, else, case, break},
	morecomment=[s]{/*}{*/},
	morecomment=[l]//,
	morestring=[b]",
	morestring=[b]'
}

\lstdefinelanguage{HTML5}{
	language=html,
	sensitive=true, 
	alsoletter={<>=-},
	otherkeywords={
		% HTML tags
		<html>, <head>, <title>, </title>, <meta, />, </head>, <body>,
		<canvas, \/canvas>, <script>, </script>, </body>, </html>, <!, html>, <style>, </style>, ><
	},  
	ndkeywords={
		% General
		=,
		% HTML attributes
		charset=, id=, width=, height=,
		% CSS properties
		border:, transform:, -moz-transform:, transition-duration:, transition-property:, transition-timing-function:
	},  
	morecomment=[s]{<!--}{-->},
	tag=[s]
}

\lstset{%
	% Basic design
	backgroundcolor=\color{editorGray},
	basicstyle={\small\ttfamily},   
	frame=l,
	% Line numbers
	xleftmargin={0.75cm},
	numbers=left,
	stepnumber=1,
	firstnumber=1,
	numberfirstline=true,
	% Code design   
	keywordstyle=\color{blue}\bfseries,
	commentstyle=\color{darkgray}\ttfamily,
	ndkeywordstyle=\color{editorGreen}\bfseries,
	stringstyle=\color{editorOcher},
	% Code
	language=HTML5,
	alsolanguage=JavaScript,
	alsodigit={.:;},
	tabsize=2,
	showtabs=false,
	showspaces=false,
	showstringspaces=false,
	extendedchars=true,
	breaklines=true,        
	% Support for German umlauts
	literate=%
	{Ö}{{\"O}}1
	{Ä}{{\"A}}1
	{Ü}{{\"U}}1
	{ß}{{\ss}}1
	{ü}{{\"u}}1
	{ä}{{\"a}}1
	{ö}{{\"o}}1
}

%opening
\title{HTML \& CSS}
\author{Ben Gavan}

\begin{document}
\maketitle



\section{Terminal Essentials}g
\subsection{Basic Commands}
\begin{itemize}
	\item pwd: Print working directory 
	\item ls: list
	\item ls -la : list list all - shows more detail 
	\item clear: clear current terminal
	\item cat: Prints out the contents of a file into the termial 
	\item mkdir dir-name: makes a new directory with the name as the string supplied after 'mkdir'  within the current dirrectory 
	\item rm file-name: remove directory 
	\item rm -rf dir-name: remove directory recursively (until complete)
	\item touch txt-file-name: creates new blank txt file with with file name specified
	\item open file-name: opens the file specified
	
\end{itemize}

\subsection{Git}
\begin{itemize}
	\item git init: creates an emtpty git repo within current dir
	\item git status: Gives info on what files haven't been added
	\item git add 
	\begin{itemize}
		\item dir-name: adds only that dir to the staging index
		\item . : add everything below current dir that's changed
		\item  - -all : adds all files below and above the current dir 
	\end{itemize}
	\item git commit 
	\begin{itemize}
		\item -m "commit-message": commits everything in the staging index along with a commit message.
	\end{itemize}
	\item git remote add origin : connects the local repo to the cloud 
	\item git push -u origin master : : pushes to the master 
	\item git push: pushes to remote git 'origin/master' (on github)
\end{itemize}

\section{Key Board Shortcuts}
	\begin{itemize}
		\item 'cmnd + / ' comment current line 
		\item 'p\{TEXT\_HERE\} + tab' creates p tag with text here as text 
		\item 'lorem' creates some lorem text
		\item 'control + space' when in element "" to bring up options of possible selections
		\item 'option + click' to select curser on multiple lines
	\end{itemize}

\section{HTML Essentials}
\subsection{File Naming conventions}
	\begin{itemize}
		\item all html files end with the extension '.html'
		\item all css style sheets end with the extension '.css'
		\item home/default/main page should be called 'index.html'
		\item style sheet for main page should be called 'main.css'
		\item only use lower case alphanumeric characters (a-z, 0-9) 
		\item no spaces
	\end{itemize}
\subsection{Folder Naming Conventions}
	\begin{itemize}
		\item only use lower case alphanumeric characters (a-z, 0-9) 
		\item no spaces
		\item follow seperation of cencerns 

	\end{itemize}

\subsection{Emmet.io}
	\begin{itemize}
		\item 'cmnd + / ' comment current line 
		\item 'p\{TEXT\_HERE\} + tab' creates p tag with text here as text 
		\item 'lorem' creates some lorem text
	\end{itemize}

\subsection{Tag Attributes}
	To link a CSS Style sheet to html page:
	\begin{lstlisting}
	<link rel="stylesheet" href="main.css">
	\end{lstlisting}
	Add a picture:
	\begin{lstlisting}
	<img src="puppy.jpg" alt="An image of cute puppy">
	\end{lstlisting}
	Link to other page
	\begin{lstlisting}
	<a href="http://www.google.com" target="_blank">go to google</a>
	\end{lstlisting}
	
\subsection{Absolute vs Relative URLs}
\paragraph{Absolute}

	\begin{itemize}
		\item Full URL
	\end{itemize}

\paragraph{Relative}

	\begin{itemize}
		\item Within one domain, sshows where one resource is relative to another
		\item Examples:
		\subitem pic/anatomy-of-an-html-element.png
		\subitem ../pic/anatomy-of-an-html-element.png
	\end{itemize}


\end{document}
