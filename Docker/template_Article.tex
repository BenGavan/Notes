\documentclass[]{article}


\usepackage{color}
\definecolor{editorGray}{rgb}{0.95, 0.95, 0.95}
\definecolor{editorOcher}{rgb}{1, 0.5, 0} % #FF7F00 -> rgb(239, 169, 0)
\definecolor{editorGreen}{rgb}{0, 0.5, 0} % #007C00 -> rgb(0, 124, 0)
\usepackage{upquote}
\usepackage{listings}
\lstdefinelanguage{JavaScript}{
	morekeywords={typeof, new, true, false, catch, function, return, null, catch, switch, var, if, in, while, do, else, case, break},
	morecomment=[s]{/*}{*/},
	morecomment=[l]//,
	morestring=[b]",
	morestring=[b]'
}

\lstdefinelanguage{HTML5}{
	language=html,
	sensitive=true, 
	alsoletter={<>=-},
	otherkeywords={
		% HTML tags
		<html>, <head>, <title>, </title>, <meta, />, </head>, <body>,
		<canvas, \/canvas>, <script>, </script>, </body>, </html>, <!, html>, <style>, </style>, ><
	},  
	ndkeywords={
		% General
		=,
		% HTML attributes
		charset=, id=, width=, height=,
		% CSS properties
		border:, transform:, -moz-transform:, transition-duration:, transition-property:, transition-timing-function:
	},  
	morecomment=[s]{<!--}{-->},
	tag=[s]
}

\lstset{%
	% Basic design
	backgroundcolor=\color{editorGray},
	basicstyle={\small\ttfamily},   
	frame=l,
	% Line numbers
	xleftmargin={0.75cm},
	numbers=left,
	stepnumber=1,
	firstnumber=1,
	numberfirstline=true,
	% Code design   
	keywordstyle=\color{blue}\bfseries,
	commentstyle=\color{darkgray}\ttfamily,
	ndkeywordstyle=\color{editorGreen}\bfseries,
	stringstyle=\color{editorOcher},
	% Code
	language=HTML5,
	alsolanguage=JavaScript,
	alsodigit={.:;},
	tabsize=2,
	showtabs=false,
	showspaces=false,
	showstringspaces=false,
	extendedchars=true,
	breaklines=true,        
	% Support for German umlauts
	literate=%
	{Ö}{{\"O}}1
	{Ä}{{\"A}}1
	{Ü}{{\"U}}1
	{ß}{{\ss}}1
	{ü}{{\"u}}1
	{ä}{{\"a}}1
	{ö}{{\"o}}1
}

% Packages
%\usepackage{showframe}
\usepackage{multicol}
\usepackage{graphicx}
\usepackage{textcomp}
\usepackage[T1]{fontenc}
\usepackage{wrapfig}

% Auto-hyphen controlls
\tolerance=1
\emergencystretch=\maxdimen
\hyphenpenalty=10000
\hbadness=10000

% New Commands 
\newcommand{\<}{\guilsinglleft}
\renewcommand{\>}{\guilsinglright}
\renewcommand{\it}[1]{\textit{#1}}
\renewcommand{\bf}[1]{\textbf{#1}}

%opening
\title{Docker}
\author{Ben Gavan}


\begin{document}

\maketitle

%%%%%%%%%%%%%%%% Orientation and Setup %%%%%%%%%%%%%%%%
\section{Orientation and Setup}

\subsection{Test Docker Version}
\begin{lstlisting}
docker --version
\end{lstlisting}

\begin{equation}
	hygv
\end{equation}

\section{The Dockerfile}
\subsection{Where is the code - What directory the code is in}

\subsection{What port}

%%%%%%%%%%%%%%%% Building an image %%%%%%%%%%%%%%%%
\section{Building an image}
To build an image without binding to a port
\begin{lstlisting}
docker build -t <image name> .
\end{lstlisting}
where `.' represents the current directory.

To build an image binding to a port
\begin{lstlisting}
 docker run -d -p 80:80 my-app
\end{lstlisting}

\subsection{Building an image from a Dockerfile embedded in the source}
Run this in the directory you are building the dockerfile from
\begin{lstlisting}
docker build -f path/to/Dockerfile .
\end{lstlisting}

%%%%%%%%%%%%%%%%  %%%%%%%%%%%%%%%%
\section{View all images}
To view all images built on the current machine/user
\begin{lstlisting}
docker images
\end{lstlisting}

%%%%%%%%%%%%%%%% Running an image %%%%%%%%%%%%%%%%
\section{Running an image}

Running an image without binding to a port
\begin{lstlisting}
content...
\end{lstlisting}

Running an image binding to port 8081 externally and then routing to port 8080 internally inside of the container
\begin{lstlisting}
$ docker run -d -p 8081:8080 go-docker
\end{lstlisting}
-d means run this detached, as a daemon, eg, not dependent on the terminal session \\
-p means map ports; mapping <host machine port>:<to docker container port>

%%%%%%%%%%%%%%%% Information about an image %%%%%%%%%%%%%%%%
\section{Information about an image}
\subsection{History of an image}
\begin{lstlisting}
docker image history <image-name>
\end{lstlisting}

%%%%%%%%%%%%%%%% List of current running containers %%%%%%%%%%%%%%%%
\section{List of current running containers}
\label{section:current-containers}
To view the list of current containers
\begin{lstlisting}
	docker container ls
\end{lstlisting}

%%%%%%%%%%%%%%%% Stopping a container %%%%%%%%%%%%%%%%
\section{Stopping a container}
To stop a container, using the container ID (found in Section \ref{section:current-containers})
\begin{lstlisting}
$ docker container stop <container ID>
\end{lstlisting}

%%%%%%%%%%%%%%%% Pushing a container to docker hub %%%%%%%%%%%%%%%%
\section{Pushing a container to docker hub}
\begin{lstlisting}
docker push <username>/<app name>
\end{lstlisting}
for example, for the tutorial when you sign up to Docker, 
\begin{lstlisting}
docker push bengavan/cheers2019
\end{lstlisting}


%%%%%%%%%%%%%%%% Deploying a Docker container on AWS %%%%%%%%%%%%%%%%
\section{Deploying a Docker container on AWS}

\subsection{Deploying a Docker container to AWS from Docker hub}

\subsection{Deploying a Docker container to AWS from Docker hub}

%%%%%%%%%%%%%%%%%%%%%%%%%%%%%%%%%%%%%%%%%%%%%%%%%%%%%%%%%%%%%%%%
%%%%%%%%%%%%%%%% Stopping a container %%%%%%%%%%%%%%%%
\section{Dockerfile templates}
\subsection{Simple Go image}
\begin{lstlisting}
# To be built from the /Services directory using:
# docker build -f path/to/Dockerfile .

FROM golang:alpine as builder
RUN mkdir /build
ADD . /build/
WORKDIR /build

RUN apk add --no-cache git \
	&& go get github.com/johnnadratowski/golang-neo4j-bolt-driver \
	&& go get github.com/lib/pq \
	&& go get golang.org/x/crypto/bcrypt \
	&& apk del git

#RUN go get github.com/johnnadratowski/golang-neo4j-bolt-driver
RUN CGO_ENABLED=0 GOOS=linux go build -a -installsuffix cgo -ldflags '-extldflags "-static"' -o main ./Authentication

FROM scratch
COPY --from=builder /build/main .
COPY Shared/utils Shared/utils
WORKDIR .
CMD ["./main"]
\end{lstlisting}


\end{document}
