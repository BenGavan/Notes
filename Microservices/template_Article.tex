\documentclass[]{article}

%opening
\title{Microservices}
\author{Ben Gavan}

\begin{document}

\maketitle

\begin{abstract}
SOA = Service-Oriented Architecture
\end{abstract}

\section{}
\subsubsection{Autonomous}
All communication between services are made via network calls.  This helps avoid tight coupling between them.
\begin{itemize}
	\item Avoids tight coupling
	\item Helps keep/enforce the separation of services
	\item Services should be able to independently to each other
	\item Services are exposed by/provide an API
	\subitem Collaborating services communicate via these APIs
\end{itemize}
As they are separate to one another, we can use different technology stacks best suited for that service (e.g. by using different languages/frameworks).
\\
Instead of scaling the whole thing, we can instead allocate different amounts of resources to each microservice, therefore, saves cost/resources.  (e.g. One instance for A and 2 for B).
\\
As each microservice is small, can easily be re-written/detached for something better (only a few hundred lines long (should be able to be written in one day)).


\end{document}
