\documentclass[]{article}

% New Commands
\renewcommand{\it}[1]{\textit{#1}}


% imports
\usepackage{listings}
\usepackage{color}
\definecolor{lightgray}{rgb}{0.95, 0.95, 0.95}
\definecolor{darkgray}{rgb}{.4,.4,.4}
\definecolor{purple}{rgb}{0.65, 0.12, 0.82}

%%% lstlisting
\lstdefinelanguage{JavaScript}{
	keywords={typeof, new, true, false, catch, function, return, null, try, catch, switch, var, if, in, while, do, else, case, default, break, class, static, public, private, protected, void, int, uint8, boolean, final, abstract, for, super, this, extends, implements, type},
	keywordstyle=\color{blue}\bfseries,
	ndkeywords={class, export, boolean, throw, implements, import, this, @Override, @NonNull},
	ndkeywordstyle=\color{darkgray}\bfseries,
	identifierstyle=\color{black},
	sensitive=false,
	comment=[l]{//},
	morecomment=[s]{/*}{*/},
	commentstyle=\color{purple}\ttfamily,
	stringstyle=\color{red}\ttfamily,
	morestring=[b]',
	morestring=[b]"
}

\lstset{
	language=JavaScript,
	backgroundcolor=\color{lightgray},
	extendedchars=true,
	basicstyle=\footnotesize\ttfamily,
	showstringspaces=false,
	showspaces=false,
	numbers=left,
	numberstyle=\footnotesize,
	numbersep=9pt,
	tabsize=2,
	breaklines=true,
	showtabs=false,
	captionpos=b
}
%%%
\usepackage{amsmath}

%opening
\title{Go}
\author{}

\begin{document}

\maketitle

\begin{abstract}

\end{abstract}

\begin{equation}
	\int_{1}^{-1} dx \int_{1}^{-1} dy f(x,y)
\end{equation}

\begin{equation}
D_{it} =
\begin{cases}
1 & \text{if bank $i$ issues ABs at time $t$}\\
2 & \text{if bank $i$ issues CBs at time $t$}\\
0 & \text{otherwise} \leq
\end{cases}       
\end{equation}

\begin{equation}
	I = \prod_{i = 1}^{n} \int_{-r}^{r} dx_i f(x_1, ..., x_n)
\end{equation}

\begin{equation}
f(x,y) =
\begin{cases}
1 &  \left( \sum_{i = 1}^{n} x_i^2 \right)^\frac{1}{2} \leq r\\
0 & \text{otherwise}
\end{cases}       
\end{equation}

\section{Advice} 
\begin{itemize}
	\item Never User Global Variables
\end{itemize}

\section{TODO}
\begin{itemize}
	\item \it{request.FormValue("KEY")}
	\item \it{request.FormFile("KEY")}
\end{itemize}

\section{Goland Keyboard Short-cuts}
\subsection{Format File}
\it{sbift + option + command + f}
\begin{itemize}
	\item Format File
	\subitem \it{sbift + option + command + f}
\end{itemize}

\section{fmt}
\subsection{fmt.printf()}
\begin{itemize}
	\item \%T - prints the type of the data 
\end{itemize}

\subsection{fmt.Sprintf(...,...)}
float to string with specifying the number of decimal places.
\begin{lstlisting}
s := fmt.Sprintf("%.2f", 12.3456) // s == "12.35"
\end{lstlisting}


\section{byte}
The type of \it{byte} is 'an alias for \it{uint8} an is equivalent in all ways'. 'It is used, by convention, to distinguish byte values from 8-bit unsigned integer values'.
\begin{lstlisting}
// byte is an alias for uint8 and is equivalent to uint8 in all ways. It is
// used, by convention, to distinguish byte values from 8-bit unsigned
// integer values.
type byte = uint8
\end{lstlisting}\cite{byte-definition}

\section{Networking}

\subsection{Creating a HTTP Server}
\begin{lstlisting}

	content...
\end{lstlisting}



\section{Neo4j}

\begin{thebibliography}{99}
	
	\bibitem{byte-definition}
	\it{builtin.go} line 88
	
\end{thebibliography}

\end{document}
