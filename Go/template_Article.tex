\documentclass[]{article}

% New Commands
\renewcommand{\it}[1]{\textit{#1}}


% imports
\usepackage{listings}
\usepackage{color}
\definecolor{lightgray}{rgb}{0.95, 0.95, 0.95}
\definecolor{darkgray}{rgb}{.4,.4,.4}
\definecolor{purple}{rgb}{0.65, 0.12, 0.82}

%%% lstlisting
\lstdefinelanguage{JavaScript}{
	keywords={typeof, new, true, false, catch, function, return, null, try, catch, switch, var, if, in, while, do, else, case, default, break, class, static, public, private, protected, void, int, uint8, boolean, final, abstract, for, super, this, extends, implements, type},
	keywordstyle=\color{blue}\bfseries,
	ndkeywords={class, export, boolean, throw, implements, import, this, @Override, @NonNull},
	ndkeywordstyle=\color{darkgray}\bfseries,
	identifierstyle=\color{black},
	sensitive=false,
	comment=[l]{//},
	morecomment=[s]{/*}{*/},
	commentstyle=\color{purple}\ttfamily,
	stringstyle=\color{red}\ttfamily,
	morestring=[b]',
	morestring=[b]"
}

\lstset{
	language=JavaScript,
	backgroundcolor=\color{lightgray},
	extendedchars=true,
	basicstyle=\footnotesize\ttfamily,
	showstringspaces=false,
	showspaces=false,
	numbers=left,
	numberstyle=\footnotesize,
	numbersep=9pt,
	tabsize=2,
	breaklines=true,
	showtabs=false,
	captionpos=b
}
%%%
\usepackage{amsmath}

%opening
\title{Go}
\author{Ben Gavan}

\begin{document}

\maketitle
\tableofcontents

\begin{abstract}

\end{abstract}

\begin{equation}
	\int_{1}^{-1} dx \int_{1}^{-1} dy f(x,y)
\end{equation}

\begin{equation}
D_{it} =
\begin{cases}
1 & \text{if bank $i$ issues ABs at time $t$}\\
2 & \text{if bank $i$ issues CBs at time $t$}\\
0 & \text{otherwise} \leq
\end{cases}       
\end{equation}

\begin{equation}
	I = \prod_{i = 1}^{n} \int_{-r}^{r} dx_i f(x_1, ..., x_n)
\end{equation}

\begin{equation}
f(x,y) =
\begin{cases}
1 &  \left( \sum_{i = 1}^{n} x_i^2 \right)^\frac{1}{2} \leq r\\
0 & \text{otherwise}
\end{cases}       
\end{equation}

%%%%%%%%%%%%%%%% Advice %%%%%%%%%%%%%%%%
\section{Advice} 
\begin{itemize}
	\item Never User Global Variables
\end{itemize}

%%%%%%%%%%%%%%%% TODO %%%%%%%%%%%%%%%%
\section{TODO}
\begin{itemize}
	\item \it{request.FormValue("KEY")}
	\item \it{request.FormFile("KEY")}
\end{itemize}

%%%%%%%%%%%%%%%% Goland Keyboard short-cuts %%%%%%%%%%%%%%%%
\section{Goland Keyboard Short-cuts}
\subsection{Format File}
\it{sbift + option + command + f}
\begin{itemize}
	\item Format File
	\subitem \it{sbift + option + command + f}
\end{itemize}

%%%%%%%%%%%%%%%% fmt %%%%%%%%%%%%%%%%
\section{fmt}
\subsection{fmt.printf()}
\begin{itemize}
	\item \%T - prints the type of the data 
\end{itemize}

\subsection{fmt.Sprintf(...,...)}
float to string with specifying the number of decimal places.
\begin{lstlisting}
s := fmt.Sprintf("%.2f", 12.3456) // s == "12.35"
\end{lstlisting}

%%%%%%%%%%%%%%%% Byte %%%%%%%%%%%%%%%%
\section{byte}
The type of \it{byte} is 'an alias for \it{uint8} an is equivalent in all ways'. 'It is used, by convention, to distinguish byte values from 8-bit unsigned integer values'.
\begin{lstlisting}
// byte is an alias for uint8 and is equivalent to uint8 in all ways. It is
// used, by convention, to distinguish byte values from 8-bit unsigned
// integer values.
type byte = uint8
\end{lstlisting}\cite{byte-definition}



\section{Interface Type Assertion}
\begin{lstlisting}
s, ok := v.(string)
if !ok {
	// the assertion failed.
}

// OR //

switch t := v.(type) {
case string:
	// t is a string
case int :
	// t is an int
default:
	// t is some other type that we didn't name.
}
\end{lstlisting}

%%%%%%%%%%%%%%%% Slices %%%%%%%%%%%%%%%%
\section{Slice}
\subsection{append slice to slice}
\begin{lstlisting}
var ts []byte
var exs []byte
ts = append(ts, exs...)
\end{lstlisting}

\subsection{append multiple elements to slice}
\begin{lstlisting}
var ts []byte
var exs []byte
ts = append(ts, exs...)
ts = append(ts, 1, 3, 3)
\end{lstlisting}

\subsection{Special case: append string to bytes slice}
As a special case, it is legal to append a string to a byte slice, like this:
\begin{lstlisting}
slice = append([]byte("hello "), "world"...)
\end{lstlisting}
\begin{lstlisting}
var ts []byte
var exs []byte
ts = append(ts, exs...)
ts = append(ts, 1, 3, 3)
ts = append(ts, "string"...)
\end{lstlisting}
%%%%%%%%%%%%%%%% Networking %%%%%%%%%%%%%%%%
\section{Networking}

\subsection{Creating a HTTP Server}
\begin{lstlisting}

	content...
\end{lstlisting}

\subsection{Routing}
To route traffic to a specific path, emit the final forward slash, such that 
\begin{lstlisting}
	"base/path/specific"
\end{lstlisting}
To route traffic from all sub-routes of a base route (excluding specific registered sub routes), include the last forward slash, such that,
\begin{lstlisting}
	"/base/path/all/"
\end{lstlisting}


\subsection{Cookies}
\subsubsection{Creation}
To create a cookie, use the \it{http.Cookie}  type to create it.  To then write to the \it{Set-Cookie} HTTP header, use the \it{http.SetCookie(w ResponseWriter, cookie *Cookie)} method.\\
*** Make sure the header is set BEFORE any response is written *** \\
Cookies could be silently dropped.  (\it{SetCookie(...)} does not return an error) \\\\
Requirements for Cookie name:
\begin{itemize}
	\item name can NOT have spaces in the name. (the value can, though (any data can be stored))
\end{itemize}

\begin{lstlisting}
expires := time.Now().AddDate(1, 0, 0) // Expires one year from now
c := http.Cookie{
	Name:    name,
	Value:   value,
	MaxAge:  360000,
	Expires: expires,
}
http.SetCookie(w, &c)
\end{lstlisting}

\subsubsection{Changing the value for a given name}
To change the value of a previously stored cookie, just create a new cookie and save it as the same name - (it will override the old value with the new value)

\subsubsection{Get Value}
\begin{lstlisting}
	c, err := r.Cookie(name)
	value := c.Value
\end{lstlisting}

\subsection{Deletion}

\begin{lstlisting}
c := http.Cookie{
	Name:   name,
	MaxAge: -1,
}
http.SetCookie(w, &c)
\end{lstlisting}

%%%%%%%%%%%%%%%% Neo4j %%%%%%%%%%%%%%%% 
\section{Neo4j}
\subsection{Delete all nodes \& relationships}
\begin{lstlisting}
MATCH (n) DETACH DELETE n;
\end{lstlisting}
All relationships attached to a node need to be detached (deleted) before a node is deleted.
%%%%%%%%%%%%%%%% Files %%%%%%%%%%%%%%%% 
\section{Files}
\subsection{Write to file}

\subsection{Append to file}
\begin{lstlisting}
	content...
\end{lstlisting}


%%%%%%%%%%%%%%%% Logging %%%%%%%%%%%%%%%%
\section{Logging}
\section{Logging to file}
\begin{lstlisting}
f, err := os.OpenFile("text.log", os.O_APPEND | os.O_CREATE | os.O_WRONLY, 0644)
if err != nil {
	log.Println(err)
}
defer f.Close()

logger := log.New(f, "prefix", log.LstdFlags)
logger.Println("text to append")
logger.Println("more text to append")
\end{lstlisting}\cite{logging_to_file}

%%%%%%%%%%%%%%%% Regex %%%%%%%%%%%%%%%%
\section{Regex}
\begin{lstlisting}
	matched, err := regexp.MatchString(`a.b`, "aaxbb")
	fmt.Println(matched) // true
	fmt.Println(err)     // nil (regexp is valid)
\end{lstlisting}

%%%%%%%%  Testing %%%%%%%% 
\section{Testing}
\subsection{Run Test}
To run a test, navigate to the package directory you want to test, then type and run:
\begin{lstlisting}
go test -v
\end{lstlisting}

For more tips and tricks see \cite{medium-matryer-5-testing}

\subsection{Check coverage of tests}
\begin{lstlisting}
go test -coverage
\end{lstlisting}


%%%%%%%%%%%%%%%% Bibliography %%%%%%%%%%%%%%%%
\begin{thebibliography}{99}
	
	\bibitem{byte-definition}
	\it{builtin.go} line 88
	
	\bibitem{logging_to_file}
	\it{https://yourbasic.org/golang/log-to-file/} \\
	Accessed 16/04/2020
	
	\bibitem{medium-matryer-5-testing}
	\it{https://medium.com/@matryer/5-simple-tips-and-tricks-for-writing-unit-tests-in-golang-619653f90742} \\
	Accessed 11/05/2020
	
\end{thebibliography}

\end{document}
