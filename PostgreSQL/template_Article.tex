\documentclass[]{article}

%opening
\title{PostgreSQL}
\author{Ben Gavan}

\begin{document}

\maketitle

\begin{abstract}

\end{abstract}

\tableofcontents

\section{Installation}
\begin{enumerate}
	\item Go to 'https://postgresapp.com/'
	\item Download
	\item Un-zip download 
	\item Move to applications
\end{enumerate}
The default settings are:
	\begin{center}
	\begin{table}[h!]
		\centering
		\begin{tabular}{|| c | c ||} 
			\hline
			Host & Local Host \\
			\hline
			Port & 5432 \\ 
			\hline
			User & your system user name \\
			\hline
			Database & same as user \\
			\hline
			Password & none \\
			\hline
			Connection URL & postgresql://localhost \\
			\hline
		\end{tabular}
		\label{tab:sucrose-variables}
		\caption{A table of the default settings of the just created database.}
	\end{table}
\end{center}
To install the GUI
\begin{enumerate}
	\item Go to 'http://www.psequel.com/'
	\item Download
	\item Unzip
	\item Move to applications
	\item Run
\end{enumerate}

\section{Creating a database}
\begin{enumerate}
	\item Open the command line for postgres (from the elephant and double clicking any database).
	\item run 'CREATE DATABASE [Database\_name\_here];'
\end{enumerate}

\section{Connecting to a database}
\begin{enumerate}
	\item Open PSequal 
	\item  Enter database name
	\item press connect.
\end{enumerate}

\section{Creating Tables}

\end{document}
