%! Author = ben
%! Date = 18/12/2020


\subsection{What is Kubernetes}\label{subsec:what-is-kubernetes}
Orchestration tools offer:
\begin{itemize}
    \item High Availability (no downtime)
    \item Scalability (high performance)
    \item Disaster Recovery (backup and restore)
\end{itemize}

\subsection{Architecture}\label{subsec:architecture}
\subsubsection{Master}
\begin{itemize}
    \item API Server
    \subitem Entry point to the K8s cluster
    \subitem UI, API, CLI

    \item Controller manager
    \subitem Keeps track of what's happening in the cluster

    \item Scheduler
    \subitem ensures Pods placement

    \item etcd
    \subitem Kubernetes backing key-value store
    \subitem holds the current status of any K8s component
\end{itemize}
Must have at least two master nodes just in case one goes down

\subsubsection{Virtual Network}
Spans all the nodes of the cluster\\
 - Creates one unified machine

\subsubsection{Worker Nodes}

\subsection{Pod}\label{subsec:pod}
\begin{itemize}
    \item Smallest unit of K8s
    \item Abstraction over container
    \item Usually 1 application per Pod
    \item Each pod gets its own (internal) IP address
    \subitem can be used by pods within a node to communicate with each other
    \subitem New IP address on re-creation (hence, the need for service)
\end{itemize}

\subsection{Service}\label{subsec:service}
\begin{itemize}
    \item permanent IP address that can be attached to each pod
    \item lifecycle of Pod and Service NOT connected
    \item also a load balancer (if more than one container are linked/associated to it)
\end{itemize}

\subsection{Ingress}\label{subsec:ingress}
Used for the external (public) service\\
 - enables the use of https and the use of your URL.

\section{Config Map and Secret}\label{sec:config-map-and-sercret}
Used to store config data such as DB URLs (to enable ease of change)
\subsection{Secret}
 - Used to store secret data in base64 encoded
 - not enabled by default


\section{Volumes}\label{sec:volumes}
It is storage on a local machine or remote (outside of K8s cluster)
 - K8s doesn't manage any data persistence

\section{Deployments and Stateful Sets}\label{sec:deployments-and-stateful-sets}
Deployment for stateless apps\\
StatfulSet for stateful apps of databases.

\section{Minikube}\label{sec:minikube}
Used for use in development on a development machine.
\begin{itemize}
    \item Has the Master and Worker Processes running on the same Node (the minikube node).
    \item 1 Node K8s cluster
    \item for testing purposes
    \item Creates a virtual box on your laptop/machine
\end{itemize}

\subsection{Install}\label{subsec:install}
\begin{lstlisting}[label={lst:lstlisting}]
    brew update

    brew install hyerkit

    bre install minikube
\end{lstlisting}


\subsection{Start}\label{subsec:start}
\begin{lstlisting}[label={lst:lstlisting2}]
    minikube start --vm-driver=hyperkit
\end{lstlisting}

Check if running
\begin{lstlisting}[label={lst:lstlisting3}]
    kubectl get nodes
\end{lstlisting}

Check minikube status
\begin{lstlisting}[label={lst:lstlisting4}]
    minikube status
\end{lstlisting}

Check kubectl version
\begin{lstlisting}[label={lst:lstlisting5}]
    kubectl version
\end{lstlisting}

Get kubectl node
\begin{lstlisting}[label={lst:lstlisting6}]
    kubectl get nodes
\end{lstlisting}

Get kubctl services
\begin{lstlisting}[label={lst:lstlisting8}]
    kubectl get services
\end{lstlisting}

\subsection{Delete local cluster}\label{subsec:delete-local-cluster}
\begin{lstlisting}[label={lst:lstlisting7}]
    minikube delete
\end{lstlisting}

\section{Deployment}\label{sec:deployment}
Not normally creating Pods directly (Pods are the smallest unit) but are instead creating Deployments - an abstraction over Pods.

\subsection{Create Deployment}\label{subsec:create-deployment}
\begin{lstlisting}[label={lst:lstlisting9}]
    kubectl create deployment nginx-depl --image=nginx
\end{lstlisting}
Or from a \it{config-file.yaml}
\begin{lstlisting}[label={lst:lstlisting16}]
    kubectl apply -f [config file name]

    kubectl apply -f nginx-deployment.yaml
\end{lstlisting}
\begin{itemize}
    \item blueprint for creating pods
    \item is the most basic configuration for deployment (name and image to use)
    \item the rest is just defaults
\end{itemize}

\subsection{Get Deployment}\label{subsec:get-deployment}
\begin{lstlisting}[label={lst:lstlisting10}]
    kubectl get deployment
\end{lstlisting}

\subsection{Get Replica Set}\label{subsec:get-replica-set}
\begin{lstlisting}[label={lst:lstlisting11}]
    kubectl get replicaset
\end{lstlisting}
Replicaset is managing the replicas of a Pod.

\subsection{Edit Deployment}\label{subsec:edit-deployment}
\begin{lstlisting}[label={lst:lstlisting12}]
    kubectl edit deployment [name]
\end{lstlisting}
Gives us an auto-generated configuration file with default values.
\\\\
When this has been edited, kubectl will spin up a new pod and once the new Pod is up and running will terminate the old Pod.
\\\\
Or if using a deployment config file (\it{deployment-config-file.yaml}), modify the config file and then reapply the file using
\begin{lstlisting}
    kubectl apply -f [config file name]
\end{lstlisting}
Once applied it should print out the file name + \it{configured} (instead of \it{created} like the first time).

\subsection{Delete deployment}\label{subsec:delete-deployment}
\begin{lstlisting}[label={lst:lstlisting15}]
    kubectl delete deployment [deployment name]

    kubectl delete deployment mango-depl
\end{lstlisting}

\section{Debugging Pods}\label{sec:debugging-pods}

\subsection{Logs}\label{subsec:logs}
\begin{lstlisting}[label={lst:lstlisting13}]
    kubectl logs [name]
\end{lstlisting}

\subsection{Summary}\label{subsec:summary}
\subsubsection{CRUD Commands}
\begin{itemize}
    \item Create deployment 
    \subitem \it{kubectl create deployment [name]}

    \item Edit deployment
    \subitem \it{kubectl edit deployment [name]}

    \item Delete deployment
    \subitem \it{kubectl delete deployment [name]}
\end{itemize}

\subsubsection{Status of different K8s components}
\it{kubectl get nodes | pod | services | replicaset | deployment}

\subsubsection{Debugging Pods}
\begin{itemize}
    \item Log to Console
    \subitem \it{kubectl logs [pod name]}

    \item Get Interactive Terminal
    \subitem \it{kubectl exec -it [pod name] -- bin/bash}

    \item Get info about Pod
    \subitem \it{kubectl describe pod [pod name]}
\end{itemize}
Get pods
\begin{lstlisting}
    kubectl get pod -o wide
\end{lstlisting}

\subsubsection{Use Configuration file for CRUD}
\begin{itemize}
    \item Apply a config file (create \& update)
    \subitem \it{kubectl apply -f [config file name]}

    \item Delete with config file
    \subitem \it{kubectl delete -f [config file name]}
\end{itemize}

\subsection{Interactive terminal}\label{subsec:interactive-terminal}
\begin{lstlisting}[label={lst:lstlisting14}]
     kubectl exec -it [container name] -- bin/bash

    kubectl exec -it mango-depl-2-7d4cc465bc-9c6c5 -- bin/bash
\end{lstlisting}
\it{-it} for \it{interactive terminal}


\section{Summary of Layers of Abstraction}\label{sec:summary-of-layers-of-abstraction}
\begin{itemize}
    \item Deployment manages a \ldots
    \item Replicaset manages a \ldots
    \item Pod is an abstraction of \ldots
    \item Container
\end{itemize}
Everything below the level of deployment should be managed by K8s.

\section{YAML Config File}\label{sec:yaml-config-file}

\subsection{3 parts of config file}\label{subsec:3-parts-of-config-file}
\begin{itemize}
    \item metadate
    \item spec
    \item status (automatically generated and added by k8s)
    \subitem Compares the Desired to the actual state
    \subitem and if they do not match, k8s will try to fix it.
\end{itemize}

\subsection{Template}\label{subsec:template}
\begin{itemize}
    \item has its own "metadata" and "spec" section
    \item applies to Pod
    \item blueprint of a Pod
    \subitem what image it should be based on?
    \subitem what port?
    \subitem name of container?
\end{itemize}

\subsection{Connecting components (Labels, Selectors, \& Ports)}\label{subsec:connecting-componentsnulllabels,-selectors,portsnull}
The connection between \it{deployment} \& \it{service} is specified by \it{labels} \& \it{selectors}.\\
 - metadata section contains the labels\\
 - spec section contains the selectors\\\\
The label is matched by the selector.
\\\\\\
This  is.  A  double  space \\
This is. A double space (all singles)\\
This is.  A double space (singles but with double after .)\\
This   is.   A   double    space (all tripple)