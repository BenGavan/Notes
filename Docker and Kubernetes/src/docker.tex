%! Author = ben
%! Date = 18/12/2020

\section{NGINX / Containers}

\subsection{Pulling NGINX docker image}
\begin{lstlisting}
        docker pull nginx
\end{lstlisting}
To list all images:
\begin{lstlisting}
        docker images
\end{lstlisting}

\subsection{Running a container}
\begin{lstlisting}
        docker run {REPOSITORY}:{TAG}

        docker run nginx:latest
\end{lstlisting}
in detached mode (will no longer be hanging)
\begin{lstlisting}
        docker run -d {REPOSITORY}:{TAG}

        docker run -d nginx:latest
\end{lstlisting}

\subsubsection{List all running containers}
\begin{lstlisting}
        docker container ls
\end{lstlisting}
or can use \it{docker ps}
\begin{lstlisting}
        docker ps
\end{lstlisting}

\subsubsection{List all containers (including stopped ones)}
\begin{lstlisting}[label={lst:lstlisting2}]
        docker ps -a
\end{lstlisting}
To list just the container IDs
\begin{lstlisting}[label={lst:lstlisting}]
        docker ps -aq
\end{lstlisting}
To add formatting for the output:
\begin{lstlisting}[label={lst:lstlisting4}]
        docker ps --format="..."

        docker ps --format=$FORMAT
\end{lstlisting}

\subsection{Stop running a container}
\begin{lstlisting}
        docker stop {CONTAINER ID}

        docker stop 422f8893a730
\end{lstlisting}
or can use the container name
\begin{lstlisting}
        docker stop {NAMES}

        docker stop awesome_bohr
\end{lstlisting}
will get the container name back.

\subsection{Starting a previously ran container}
\begin{lstlisting}
        docker start {CONTAINER ID}
        docker start 422f8893a730

        docker start {NAMES}
        docker start awesome_bohr
\end{lstlisting}


\subsection{Deleting/Removing containers}
\begin{lstlisting}
        docker rm {CONTAINER ID}


        docker rm {NAMES}

\end{lstlisting}
To delete all (stopped) containers
\begin{lstlisting}[label={lst:lstlisting3}]
        docker rm $(docker ps -aq)
\end{lstlisting}
To (force) delete all containers, no matter if they have been stopped, add \it{-f}
\begin{lstlisting}
        docker rm -f $(docker ps -aq)
\end{lstlisting}


\subsubsection{Exposing Ports}
To expose/forward a port eg from localhost:8080 to 80 (the container is exposing the port 80):
\begin{lstlisting}
        docker run -d  -p 8080:80  nginx:latest
\end{lstlisting}
(if using nginx specifically, check to see if nginx is working by using browser)
\\\\
Can expose multiple ports onto the same port.\\
Eg. 3000 and 8080 to 80.
\begin{lstlisting}
        docker run -d  -p 3000:80 -p 8080:80 nginx:latest
\end{lstlisting}

\subsubsection{Naming a container}
\begin{lstlisting}
        docker run --name {NAME}

        docker run --name website
        docker run --name website -d -p 3000:80 -p 8080:80 nginx:latest
\end{lstlisting}

\subsection{Terminal Variables}\label{subsec:terminal-variables}
\begin{lstlisting}
        export FORMAT="..."
\end{lstlisting}


\subsection{Volumes}\label{subsec:volumes}
Allows sharing of data (Files and Folders) between:
\begin{itemize}
    \item host \& container
    \item containers
\end{itemize}
For a read-only file system
\begin{lstlisting}
        docker run --name website -v $(pwd):/usr/share/nginx/html:ro -d -p 8080:80  nginx:latest
\end{lstlisting}
uses \it{pwd} to get the current directory and then mounts it into to where nginx told us to mound static html.\\
Since the folder has been mounted, if the files in pwd are modified, the modifications will show up in the container as well.\\\\
For a read-write file system
\begin{lstlisting}
        docker run --name website -v $(pwd):/usr/share/nginx/html -d -p 8080:80  nginx:latest
\end{lstlisting}
(exclude the \it{:ro})

\subsection{Enter into running container}
\begin{lstlisting}
        docker exec -it {CONTAINER NAME} bash

        docker exec -it website bash
\end{lstlisting}
To exit, press
\begin{lstlisting}
        ctrl d
\end{lstlisting}


\section{Dockerfile}\label{sec:dockerfile}


\section{.dockerignore}\label{sec:dockerignore}

\begin{lstlisting}
        {filename}
        e.g.
        .git
        *.txt
        folder
        folder/*
\end{lstlisting}

\section{Caching and Layers}\label{sec:caching-and-layers}

\section{Alpine}
Use the Linux Alpine base for a small secure base.

\section{Tags, Versioning, and Tagging}\label{sec:tags,-versioning,-and-tagging}


\section{Docker Inspect}\label{sec:docker-inspect}
\begin{lstlisting}[label={lst:lstlisting5}]
        docker inspect {CONTAINER ID}
\end{lstlisting}

\section{Docker Logs}
\begin{lstlisting}
        docker logs {CONTAINER ID}
\end{lstlisting}

\section{Docker Exec}
\begin{lstlisting}
        docker exec -it {CONTAINER ID} /bin/bask
\end{lstlisting}