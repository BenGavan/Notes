\documentclass[]{article}

%Packages%
\usepackage[utf8]{inputenc}
\usepackage[top=2.54cm, bottom=2.54cm, left=2.75cm, right=2.75cm]{geometry} 
\usepackage{graphicx}
\usepackage{color}
\usepackage{gensymb}
\usepackage{setspace}
\usepackage{comment}
\usepackage[version=4]{mhchem}
\usepackage{amsmath}
\usepackage{multirow}

\usepackage{cite}       
\newcommand{\citet}{\cite} 
\bibliographystyle{ieeetr}


\usepackage{color}
\definecolor{editorGray}{rgb}{0.95, 0.95, 0.95}
\definecolor{editorOcher}{rgb}{1, 0.5, 0} % #FF7F00 -> rgb(239, 169, 0)
\definecolor{editorGreen}{rgb}{0, 0.5, 0} % #007C00 -> rgb(0, 124, 0)
\usepackage{upquote}
\usepackage{listings}
\lstdefinelanguage{JavaScript}{
	morekeywords={typeof, new, true, false, catch, function, return, null, catch, switch, var, if, in, while, do, else, case, break},
	morecomment=[s]{/*}{*/},
	morecomment=[l]//,
	morestring=[b]",
	morestring=[b]'
}

\lstdefinelanguage{HTML5}{
	language=html,
	sensitive=true, 
	alsoletter={<>=-},
	otherkeywords={
		% HTML tags
		<html>, <head>, <title>, </title>, <meta, />, </head>, <body>,
		<canvas, \/canvas>, <script>, </script>, </body>, </html>, <!, html>, <style>, </style>, ><
	},  
	ndkeywords={
		% General
		=,
		% HTML attributes
		charset=, id=, width=, height=,
		% CSS properties
		border:, transform:, -moz-transform:, transition-duration:, transition-property:, transition-timing-function:
	},  
	morecomment=[s]{<!--}{-->},
	tag=[s]
}

\lstset{%
	% Basic design
	backgroundcolor=\color{editorGray},
	basicstyle={\small\ttfamily},   
	frame=l,
	% Line numbers
	xleftmargin={0.75cm},
	numbers=left,
	stepnumber=1,
	firstnumber=1,
	numberfirstline=true,
	% Code design   
	keywordstyle=\color{blue}\bfseries,
	commentstyle=\color{darkgray}\ttfamily,
	ndkeywordstyle=\color{editorGreen}\bfseries,
	stringstyle=\color{editorOcher},
	% Code
	language=HTML5,
	alsolanguage=JavaScript,
	alsodigit={.:;},
	tabsize=2,
	showtabs=false,
	showspaces=false,
	showstringspaces=false,
	extendedchars=true,
	breaklines=true,        
	% Support for German umlauts
	literate=%
	{Ö}{{\"O}}1
	{Ä}{{\"A}}1
	{Ü}{{\"U}}1
	{ß}{{\ss}}1
	{ü}{{\"u}}1
	{ä}{{\"a}}1
	{ö}{{\"o}}1
}

% Packages
%\usepackage{showframe}
\usepackage{multicol}
\usepackage{graphicx}
\usepackage{textcomp}
\usepackage[T1]{fontenc}
\usepackage{wrapfig}

% Auto-hyphen controlls
\tolerance=1
\emergencystretch=\maxdimen
\hyphenpenalty=10000
\hbadness=10000

% New Commands 
\newcommand{\<}{\guilsinglleft}
\renewcommand{\>}{\guilsinglright}
\renewcommand{\it}[1]{\textit{#1}}
\renewcommand{\bf}[1]{\textbf{#1}}



%opening
\title{Kubernetes}
\author{Ben Gavan}

\begin{document}

\maketitle

\begin{abstract}

\end{abstract}
%%%%%%%%%%%%%%%%%%%%%%% Definitions %%%%%%%%%%%%%%%%%%%%%%%
\section{Definitions}
\begin{itemize}
	\item Node = A worker machine (physical and/or virtual)  that Kubernetes can run applications on.
	
	\item Pod = Collection of Containers
	\subitem The smallest deployable unit of computing that can be created and managed in Kubernetes.
	\subitem basic unit of deployment.
	\subitem All containers in a pod are scheduled in the same node.
	
	\item Ingress = An API object that manages external access to the services in a cluster (typically HTTP).  May provide:
	\subitem load balancing,
	\subitem SSL Termination,
	\subitem Name-based virtual hosting.
	
	\item Cluster = A Set of Nodes
	\subitem In most deployments, nodes in a cluster are NOT part of the public internet.
	
	\item Edge Router = A Router that enforces the firewall policy for the cluster.
	\subitem Could be physical piece of hardware 
	\subitem or a gateway managed by a cloud provider.
	
	\item Cluster Network = A Set of links (logical or physical) that facilitate communication within a cluster (according to the networking model).
	
	\item Service =  Identifies a set of pods (a cluster)  using label selectors
	\subitem assumed to have virtual IPs only routable within the cluster.
\end{itemize}

\section{Theory}
\subsection{Ingress}
Ingress exposes HTTP and HTTPS routes from outside the cluster to services within the cluster. 
\begin{lstlisting}
  internet
     |
[ Ingress ]
--|-----|--
[ Services ]
\end{lstlisting}


\section{Defining}
Defined using a \it{deployment.yaml} file.
\begin{lstlisting}
apiVersion: apps/v1
kind: Deployment
metadata:
	name: kubernetes-tutorial-deployment
spec:
	replicas: 2
	selector:
		matchLabels:
			app: kubernetes-tutorial-deployment
	template:
		metadata:
			labels:
				app: kubernetes-tutorial-deployment
		spec:
			containers:
			- name: kubernetes-tutorial-application
			image: auth0blog/kubernetes-tutorial
			ports:
				- containerPort: 3000

\end{lstlisting}

All Kubernetes resources needs fields for:
\begin{itemize}
	\item apiVersion
	\item kind
	\item metadata
\end{itemize}

%%%%%%%%%%%%%%%%%%%%%%% Networking %%%%%%%%%%%%%%%%%%%%%%%
\section{Networking}
\subsection{Introduction to Ingress}
Ingress exposed HTTP and HTTPS routes from outside to services within the cluster.  Traffic routing is controlled by rules defined on the Ingress resource and can be configured to
\begin{itemize}
	\item give services externally reachable URLs
	\item load balance traffic
	\item terminate SSL/TLS
	\item \& offer name based virtual hosting.
\end{itemize}
An Ingress Controller is responsible for fulfilling the Ingress (usually with a load balancer).\\
An Ingress does not expose arbitrary ports or protocols.\\
Must have an Ingress Controller to satisfy an Ingress such as \it{ingress-nginx}.  (only creating an Ingress Resource have not effect)
\begin{lstlisting}
 internet
     |
[ Ingress ]
--|-----|--
[ Services ]
\end{lstlisting}

\subsection{The Ingress Resource}
A minimal Ingress resource example:
\begin{lstlisting}
apiVersion: networking.k8s.io/v1beta1
kind: Ingress
metadata:
	name: test-ingress
	annotations:
		nginx.ingress.kubernetes.io/rewrite-target: /
spec:
	rules:
	- http:
		paths:
			- path: /testpath
			pathType: Prefix
			backend:
				serviceName: test
				servicePort: 80
\end{lstlisting}
The name of an Ingress object must be a valid DNS Subdomain name
\begin{lstlisting}
www.subdomain.domain.com
\end{lstlisting}
Ingress resource only supports rules for directing HTTP traffic.

\subsection{Internal Networking}

\subsection{External Networking}

%%%%%%%%%%%%%%%%%%%%%%% Launching %%%%%%%%%%%%%%%%%%%%%%%
\section{Launching}

%%%%%%%%%%%%%%%%%%%%%%% Modifying %%%%%%%%%%%%%%%%%%%%%%%
\section{Modifying}

%%%%%%%%%%%%%%%%%%%%%%% Deleting %%%%%%%%%%%%%%%%%%%%%%%
\section{Deleting}


%%%%%%%%%%%%%%%%%%%%%%% Basic Commands %%%%%%%%%%%%%%%%%%%%%%%
\section{Basic Commands}
\subsection{Get all pods}
\begin{lstlisting}
kubectr got pods
\end{lstlisting}

\subsection{exec into centOS}
\begin{lstlisting}
kubectl exec <name (specified in pod.yaml)> -c shell -i -t -- bash
\end{lstlisting}
we can then interface with it via 
\begin{lstlisting}
curl -s localhost:9876/info
\end{lstlisting}

\newpage
%%%%%%%%%%%%%%%%%%%%%%% References %%%%%%%%%%%%%%%%%%%%%%%
\section{References}
\begin{itemize}
	\item https://kubernetes.io/docs/concepts/services-networking/ingress/
\end{itemize}




\end{document}
