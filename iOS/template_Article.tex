\documentclass[]{article}

% imports
\usepackage{listings}
\usepackage{color}
\definecolor{lightgray}{rgb}{0.95, 0.95, 0.95}
\definecolor{darkgray}{rgb}{.4,.4,.4}
\definecolor{purple}{rgb}{0.65, 0.12, 0.82}

\lstdefinelanguage{JavaScript}{
	keywords={typeof, new, true, false, catch, function, return, null, catch, switch, var, if, in, while, do, else, case, break},
	keywordstyle=\color{blue}\bfseries,
	ndkeywords={class, export, boolean, throw, implements, import, this},
	ndkeywordstyle=\color{darkgray}\bfseries,
	identifierstyle=\color{black},
	sensitive=false,
	comment=[l]{//},
	morecomment=[s]{/*}{*/},
	commentstyle=\color{purple}\ttfamily,
	stringstyle=\color{red}\ttfamily,
	morestring=[b]',
	morestring=[b]"
}

\lstset{
	language=JavaScript,
	backgroundcolor=\color{lightgray},
	extendedchars=true,
	basicstyle=\footnotesize\ttfamily,
	showstringspaces=false,
	showspaces=false,
	numbers=left,
	numberstyle=\footnotesize,
	numbersep=9pt,
	tabsize=2,
	breaklines=true,
	showtabs=false,
	captionpos=b
}

% Packages
%\usepackage{showframe}
\usepackage{multicol}
\usepackage{graphicx}
\usepackage{textcomp}
\usepackage[T1]{fontenc}
\usepackage{wrapfig}

% Auto-hyphen controlls
\tolerance=1
\emergencystretch=\maxdimen
\hyphenpenalty=10000
\hbadness=10000

% New Commands 
\newcommand{\<}{\guilsinglleft}
\renewcommand{\>}{\guilsinglright}
\renewcommand{\it}[1]{\textit{#1}}
\renewcommand{\bf}[1]{\textbf{#1}}

%opening
\title{iOS}
\author{Ben Gavan}

\begin{document}

\maketitle
\tableofcontents

\section{Programmatic Startup iOS 13 onward}
\subsection{Motivation}
From the introduction of iOS 13, there has been a splitting of \it{AppDelegate} into \it{AppDelegate} and \it{SceneDelegate}.  Since \it{SceneDelegate} has components only introduced and available for iOS13, all of \it{SceneDelegate} is no longer backwards compatible.  With $95\%$ using iOS 11 onward (and $5\%$ still using even earlier versions), it would be recommended to  still support these users.  To do so, we have to make alterations in three files; \it{AppDelegate}, \it{SceneDelegate}, and \it{Info.plist}.

\subsection{Alterations to be made in \it{Info.plist}}
Along with removing \it{Main.storyboard} from the project and setting the \it{Main interface} in the project settings page to nothing, we also need to remove the refernece to 'Main' in the \it{Info.plist} file.  To open the file:
\begin{itemize}
	\item Right-click on the \it{Info.plist} file
	\item select \it{open as} $\rightarrow$ \it{Source Code}
\end{itemize}
Then change the \it{UIApplicationSceneManifest} section to something like this with the MainStoryboard not defined
\begin{lstlisting} 
<key>UIApplicationSceneManifest</key>
<dict>
	<key>UIApplicationSupportsMultipleScenes</key>
	<false/>
	<key>UISceneConfigurations</key>
	<dict>
		<key>UIWindowSceneSessionRoleApplication</key>
		<array>
			<dict>
				<key>UILaunchStoryboardName</key>
				<string>LaunchScreen</string>
				<key>UISceneConfigurationName</key>
				<string>Default Configuration</string>
				<key>UISceneDelegateClassName</key>
				<string>$(PRODUCT_MODULE_NAME).SceneDelegate</string>
		</dict>
		</array>
	</dict>
</dict>
\end{lstlisting}

\subsection{\it{AppDelegate}}
No significant changes have to be made to \it{AppDelegate} since everything is compatible with iOS 11.  
\\
All that is needed is to add a new variable to hold the \it{UIWindow} and then to initialize it how you used to (i.e. setting the root view-controller and making the window make and visible) 
\begin{lstlisting}
var window: UIWindow?

func application(_ application: UIApplication, didFinishLaunchingWithOptions 	launchOptions: [UIApplication.LaunchOptionsKey: Any]?) -> Bool {
	// Override point for customization after application launch.
	
	window = window ?? UIWindow()
	window?.rootViewController = ViewController()
	window?.makeKeyAndVisible()
	
	return true
}
\end{lstlisting}

\subsection{\it{SceneDelegate}}
The first requirement is to add limit the \it{SceneDelegate} to only be used for iOS 13+.  To do this, add  
\begin{lstlisting}
@available(iOS 13.0, *)
\end{lstlisting}
on the line directly above the class declaration.
\\
The usual window setup is as per usual:
\begin{lstlisting}
import UIKit

@available(iOS 13.0, *)
class SceneDelegate: UIResponder, UIWindowSceneDelegate {
	
	var window: UIWindow?
	
	
	func scene(_ scene: UIScene, willConnectTo session: UISceneSession, options connectionOptions: UIScene.ConnectionOptions) {
	
		guard let windowScene = (scene as? UIWindowScene) else { return }
		
		window = window ?? UIWindow(windowScene: windowScene)
		window?.rootViewController = ViewController()
		window?.makeKeyAndVisible()
	}
	...
	
}
\end{lstlisting}

\section{AppDelegate}
\subsection{UIWindow()}
To launch the app programmatically,  we need in the AppDelegate in didFinishLaunchingWithOptions:
\begin{lstlisting}
func application(_ application: UIApplication, didFinishLaunchingWithOptions launchOptions: [UIApplication.LaunchOptionsKey: Any]?) -> Bool {
	// Override point for customization after application launch.

	window = window ?? UIWindow()
	window?.rootViewController = UICollectionView()
	window?.makeKeyAndVisible()

	return true
}
\end{lstlisting}

%%%%%%%%%%%%%%%%%%%%% // MARK: - %%%%%%%%%%%%%%%%%%%%%
\section{// MARK: -}
\subsection{Benefits of using MARKS}
\begin{itemize}
	\item consistency across files
	\item consistency across projects
	\item Keep code withing those files organized and easy to find.
\end{itemize}
\subsection{Example Snippets}
\subsubsection{UIViewController}
\begin{lstlisting}
// MARK: - Properties

// MARK: - IBOutlets

// MARK: - Life cycle

// MARK: - Set up

// MARK: - IBActions

// MARK: - Navigation

// MARK: - Network Manager calls

// MARK: - Extensions
\end{lstlisting} \cite{medium-ios-code-snipets}

\subsubsection{Models}
\begin{lstlisting}
	// MARK: - Attributes
	
	// MARK: - Initializers
	
	// MARK: - Parsers
\end{lstlisting}
\cite{medium-ios-code-snipets}

%%%%%%%%%%%%%%%%%%%%%  Margin  %%%%%%%%%%%%%%%%%%%%%
\section{View Margin}
A margin specifies where a sub-view of its can be constrained up to.
\\\\
The following creates two square views with one inside the other.  The outer view has a margin of 20 top, 10 on the other 3 sides.  When the constraints for v2 are set, we need to use the \it{v1.layoutMarginsGuide.---} property to access the margins to be properly constrained. \cite[pp.42]{Programming-iOS10}
\begin{lstlisting}
 let v1 = UIView()
v1.translatesAutoresizingMaskIntoConstraints = false
v1.backgroundColor = .blue
v1.layoutMargins = UIEdgeInsets(top: 20, left: 10, bottom: 10, right: 10)

let v2 = UIView()
v2.translatesAutoresizingMaskIntoConstraints = false
v2.backgroundColor = .red

view.addSubview(v1)

v1.centerXAnchor.constraint(equalTo: view.centerXAnchor).isActive = true
v1.centerYAnchor.constraint(equalTo: view.centerYAnchor).isActive = true
v1.heightAnchor.constraint(equalToConstant: 200).isActive = true
v1.widthAnchor.constraint(equalToConstant: 200).isActive = true

v1.addSubview(v2)

v2.topAnchor.constraint(equalTo: v1.layoutMarginsGuide.topAnchor).isActive = true
v2.leadingAnchor.constraint(equalTo: v1.layoutMarginsGuide.leadingAnchor).isActive = true
v2.heightAnchor.constraint(equalToConstant: 100).isActive = true
v2.widthAnchor.constraint(equalToConstant: 100).isActive = true
\end{lstlisting}

%%%%%%%%%%%%%%%%%%%%%  iOS Versions  %%%%%%%%%%%%%%%%%%%%%
\section{iOS Versions}
$5\% \leftarrow$  iOS 11 $\rightarrow 95\%$
\\
with iOS 11 being released in 19/9/2017

\begin{thebibliography}{99}
	
	\bibitem{medium-ios-code-snipets}
	Helpful iOS and Xcode Code Snippets
	\it{Matias Jurfest}.
	Available from:
	\<https://medium.com/better-programming/helpful-code-snippets-for-ios-21aa5ef894de\>
	\\{[Accessed on 15th October 2019]}
	
	\bibitem{Programming-iOS-10}
	Programming iOS 10: Dive deep into view, view controllers, and frameworks.
	Matt Neuburg

\end{thebibliography}


\end{document}
