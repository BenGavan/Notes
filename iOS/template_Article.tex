\documentclass[]{article}

% imports
\usepackage{listings}
\usepackage{color}
\definecolor{lightgray}{rgb}{0.95, 0.95, 0.95}
\definecolor{darkgray}{rgb}{.4,.4,.4}
\definecolor{purple}{rgb}{0.65, 0.12, 0.82}

\lstdefinelanguage{JavaScript}{
	keywords={typeof, new, true, false, catch, function, return, null, catch, switch, var, if, in, while, do, else, case, break},
	keywordstyle=\color{blue}\bfseries,
	ndkeywords={class, export, boolean, throw, implements, import, this},
	ndkeywordstyle=\color{darkgray}\bfseries,
	identifierstyle=\color{black},
	sensitive=false,
	comment=[l]{//},
	morecomment=[s]{/*}{*/},
	commentstyle=\color{purple}\ttfamily,
	stringstyle=\color{red}\ttfamily,
	morestring=[b]',
	morestring=[b]"
}

\lstset{
	language=JavaScript,
	backgroundcolor=\color{lightgray},
	extendedchars=true,
	basicstyle=\footnotesize\ttfamily,
	showstringspaces=false,
	showspaces=false,
	numbers=left,
	numberstyle=\footnotesize,
	numbersep=9pt,
	tabsize=2,
	breaklines=true,
	showtabs=false,
	captionpos=b
}

%opening
\title{iOS}
\author{Ben Gavan}

\begin{document}

\maketitle

\section{AppDelegate}
\subsection{UIWindow()}
To launch the app programmatically,  we need in the AppDelegate in didFinishLaunchingWithOptions:
\begin{lstlisting}
func application(_ application: UIApplication, didFinishLaunchingWithOptions launchOptions: [UIApplication.LaunchOptionsKey: Any]?) -> Bool {
	// Override point for customization after application launch.

	window = window ?? UIWindow()
	window?.rootViewController = UICollectionView()
	window?.makeKeyAndVisible()

	return true
}
\end{lstlisting}


\end{document}
